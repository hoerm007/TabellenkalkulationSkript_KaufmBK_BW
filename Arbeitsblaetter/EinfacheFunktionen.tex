% !TeX root = ../Skript_Tabellen.tex
\cohead{\Large\textbf{Einfache Funktionen}}
\subsection{Einfache Funktionen}
Sie haben bereits gelernt, wie man Daten erfassen und formatieren kann. Bei Calc handelt es sich um ein Tabellenkalkulationsprogramm, der typische Verwendungszweck ist also nicht nur das Erfassen und Darstellen von Daten, sondern die Verarbeitung von Daten. Um eine Kalkulation oder eine Berechnung durchzuführen, wählen Sie eine leere Zelle aus und beginnen Sie mit einem Gleichzeichen.
\begin{tcolorbox}
    Alle Berechnungen beginnen immer mit einem Gleichzeichen:

    \lstinline|=BERECHNUNG|
\end{tcolorbox}
\Calc unterstützt unter anderem folgende Rechenoperationen und hält sich an die normale Reihenfolge der Rechengesetze, also Klammern, Potenzen, dann Punkt-vor-Strich:
\begin{itemize}
    \item + Addition
    \item - Subtraktion
    \item * Multiplikation
    \item / Division
    \item () Klammern
    \item \(\wedge\) Potenzen
\end{itemize}
\begin{Exercise}[title={Öffnen Sie Calc und\dots}, label=EinfacheBerechnungen]
    \begin{enumerate}
        \item geben Sie in \zelle{A1} \lstinline|=1+2| ein und drücken Sie dann Enter.

        Was zeigt \Calc an?

        Markieren Sie die Zelle. Was zeigt die Eingabezeile an?
        \item Geben Sie in \zelle{A2} \lstinline|=3*4| ein und drücken Sie dann Enter.
        \item Geben Sie in \zelle{B1} \lstinline|=2+3*5| ein und drücken Sie dann Enter.

        Warum ist das Ergebnis nicht 25? Schließlich ist 2+3 gleich 5 und 5*5 ist 25.
    \end{enumerate}
\end{Exercise}
\begin{Answer}[ref=EinfacheBerechnungen]
    \begin{enumerate}
        \item Die Zelle zeigt das Ergebnis, also 3, an. Die Eingabezeile zeigt die Eingabe also \lstinline|=1+2| an:

        \includegraphics*[width=.4\textwidth]{\pics/1Plus2.png}
        \item Die Zelle zeigt 12 an, die Eingabezeile \lstinline|=3*4|.
        \item Die Zelle zeigt 17 an, da \Calc Punkt-vor-Strich beachtet und daher zuerst 3*5 gleich 15 rechnet und dann 2 dazu addiert.
    \end{enumerate}
\end{Answer}