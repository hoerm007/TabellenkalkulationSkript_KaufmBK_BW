% !TeX root = ../Skript_Tabellen.tex
\cohead{\Large\textbf{Einfache Berechnungen}}
\subsection{Einfache Berechnungen}
Sie haben bereits gelernt, wie man Daten erfassen und formatieren kann. Bei Calc handelt es sich um ein Tabellenkalkulationsprogramm, der typische Verwendungszweck ist also nicht nur das Erfassen und Darstellen von Daten, sondern die Verarbeitung von Daten. Um eine Kalkulation oder eine Berechnung durchzuführen, wählen Sie eine leer Zelle aus und beginnen Sie mit einem Gleichheitszeichen.
\begin{tcolorbox}
    Alle Berechnungen beginnen immer mit einem Gleichzeichen:

    \lstinline|=BERECHNUNG|
\end{tcolorbox}
\begin{Exercise}[title={Öffnen Sie Calc und\dots}, label=EinfacheBerechnungen]
    \begin{enumerate}
        \item geben Sie in \zelle{A1} \lstinline|=1+2| ein.
    \end{enumerate}
\end{Exercise}


\href{file:./Uebungsblaetter/2_EinfacheBerechnungen.ods}{text}




%%%%%%%%%%%%%%%%%%%%%%%%%%%%%%%%%%%%%%%%%
\begin{Answer}[ref=EinfacheBerechnungen]

\end{Answer}